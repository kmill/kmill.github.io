\documentclass[11pt,letterpaper,roman]{moderncv}

\usepackage{textpos}

\moderncvstyle{classic}
%\moderncvstyle{casual}
\moderncvcolor{blue}
\usepackage[utf8]{inputenc}

%\nopagenumbers{}

\usepackage[margin=0.75in]{geometry}
\recomputelengths

\firstname{Kyle}
\lastname{Miller}
\email{kymiller@ucsc.edu}
\address{4167 McHenry Library}{Santa Cruz, CA 95064}
\homepage{kmill.github.io}

\begin{document}
\maketitle
\vspace{-2em}

\section{Academic Appointments}

\cventry{2021--present}{Postdoc}{Department of Mathematics, University of California}{Santa Cruz, CA}{}{}

\cventry{2022--2023}{Postdoc}{Laboratoire de Mathématiques d'Orsay, Universit\'{e} Paris-Saclay}{Orsay, France}{}{}

\section{Research Interests}

I am interested in finding ways that computers can be used in mathematics, education, and engineering.
I work on interactive theorem provers, improving them and exploring applications of this technology beyond formal verification.

\vspace{0.5em}
My mathematics research centers around low-dimensional topology, knot theory, and singularity theory.


\section{Education}

\cventry{2014--2022}{Ph.D.}{University of California}{Berkeley, CA}{}
{Advisor: Ian Agol.\\
Thesis: \emph{Singularity theory for extended cobordism categories and an application to graph theory}.}

\cventry{2008--2012}{S.B.}{Massachusetts Institute of Technology}{Cambridge, MA}{}
{Major: Mathematics with Computer Science. Minor: Music.}

\section{Professional Experience}

\cventry{2023--}{Research Software Engineer}{Lean Focused Research Organization (FRO)}{}{}{
I look for, specify, and implement improvements to Lean, the functional programming language and interactive theorem prover.
}

\cventry{Su2015}{Software Engineer}{Swift Navigation, Inc.}{San Francisco, CA}{}{
Designed and implemented \emph{Plover}, an experimental programming language for linear algebra in embedded applications, with Scott Kovach.
}

\cventry{2013--2014}{Research Assistant}{Microsoft Research New England}{Cambridge, MA}{}{
Empirical microeconomics research with Markus Mobius and Susan Athey regarding news bias in social media.
Designed analyses to run efficiently on hundreds of terabytes of data.
%(Aided Greg Minton in discovering a symmetric periodic solution to the $4$-body problem.)
}

\cventry{2012--2013}{Software Engineer}{Vecna Technologies, Inc.}{Cambridge, MA}{}{
Enterprise Java software for online healthcare systems.
}


\section{Publications and Preprints}

\subsection{In preparation}

\cvitem{}{Kolichala, Kovach, Miller, and Kjolstad, \emph{Indexed stream fusion: a compiler framework for optimizing traversals over general container types}}

\cvitem{}{Miller and Massot, \emph{Informalization: natural structured proofs from formalized mathematics}}

%\cvitem{}{Miller, \emph{Doing graph theory in Lean}}

\cvitem{}{Miller, \emph{The two-variable virtual Yamada polynomial}.}

\cvitem{}{Miller, \emph{Surface graph invariants as extended 2D TQFTs}.}

%\cvitem{}{\emph{A category in graph theory}.}

\subsection{Published}

\cvitem{2023}{Miller, \emph{The homological arrow polynomial for virtual links}, Journal of Knot Theory and Its Ramifications
(2023), \href{https://doi.org/10.1142/S0218216523500050}{doi:10.1142/S0218216523500050}.}

\cvitem{2021}{Anderson, Baker, Gao, Kegel, Le, Miller, Onaran, Sangston, Tripp, Wood, and Wright, \emph{L-space knots with tunnel number $>1$ by experiment}, Experimental Mathematics (2021), \href{https://doi.org/10.1080/10586458.2021.1980753}{doi:10.1080/10586458.2021.1980753}.}

\cvitem{2018}{McPhail-Snyder and Miller, \emph{Planar diagrams for local invariants of graphs in surfaces}, Journal of Knot Theory and Its Ramifications
  (2020), \href{https://doi.org/10.1142/S0218216519500937}{doi:10.1142/S0218216519500937}.}

\subsection{Preprints}

\cvitem{2020}{Gusakov, Mehta and Miller, \emph{Formalizing Hall's Marriage Theorem in Lean}. \href{https://arxiv.org/abs/2101.00127}{arXiv:2101.00127}.}

\cvitem{2020}{Miller, \emph{All the ways I know how to define the Alexander Polynomial} \href{https://math.berkeley.edu/~kmill/notes/alexander_poly_2020_11_14.pdf}{(link to pdf)}}

\section{Open-source Artifacts}

\cvitem{2019--present}{\textbf{KnotFolio}, an online program for recognizing and identifying drawings of knots and links. \href{https://knotfol.io/}{https://knotfol.io/}}
\cvitem{2021--present}{\textbf{Mathlib}, the Lean mathematics library, maintainer and contributor: \href{https://leanprover-community.github.io/teams/maintainers.html}{https://leanprover-community.github.io/teams/maintainers.html}}
\cvitem{2021--present}{\textbf{Pyquiz}, a tool for constructing Canvas quizzes with randomization and custom explanations, with modules for linear algebra. \href{https://github.com/UCBMath/pyquiz}{https://github.com/UCBMath/pyquiz}}
\cvitem{2018--present}{\textbf{Planalg}, a Mathematica library for computations with planar algebras and diagrammatic categories. \href{https://github.com/kmill/planalg}{https://github.com/kmill/planalg}}
\cvitem{2015}{\textbf{Plover}, high-level programming language for linear algebra on embedded systems. \href{https://github.com/swift-nav/plover}{https://github.com/swift-nav/plover}}

\section{Talks}

\subsection{Invited}
\cvitem{Jan 2024}{UCSC CSE Colloquium. \emph{To formalized mathematics and back with the Lean theorem prover}.}
\cvitem{Jan 2024}{Special Session on Algebraic Structures in Knot Theory. \emph{The homological arrow polynomial for virtual links}.}
\cvitem{Sep 2023}{Workshop on Libraries of Formal Proofs and Natural Mathematical Language, EuroProofNet. \emph{Informalizing formalized mathematics using the Lean theorem prover}.}
\cvitem{Apr 2023}{University of Frieburg algebra seminar. \emph{Informalizing formalized mathematics using the Lean theorem prover}.}
\cvitem{Apr 2023}{Languages, Systems, and Data Seminar. \emph{Informalizing formalized mathematics using the Lean theorem prover}.}
\cvitem{Nov 2022}{Universit\'{e} Paris-Saclay seminar on computer formalization of mathematics. \emph{Some thoughts on formalizing basic knot theory}.}
\cvitem{Nov 2021}{UC Santa Cruz geometry and analysis seminar. \emph{The homological arrow polynomial}.}
\cvitem{Nov 2021}{Oklahoma State University topology seminar. \emph{The homological arrow polynomial}.}
\cvitem{Jan 2021}{Special Session on Developments in Spatial Graphs, JMM. \emph{A 2D TQFT approach to topological graph polynomials and graphs in thickened surfaces.}}
\cvitem{Dec 2019}{University of Virginia geometry seminar. \emph{A TQFT approach to topological graph polynomials}.}
\cvitem{Nov 2019}{Rice topology seminar. \emph{Invariants of graphs in thickened surfaces from topological graph polynomials}.}
\cvitem{Nov 2019}{Special Session on Invariants of Knots and Spatial Graphs, Fall Western Sectional Meeting of the AMS. \emph{Invariants of virtual spatial graphs based on topological graph polynomials}.}

% \subsection{Research}

% \cvitem{Apr 2018}{$3$-manifold seminar, UCB. \emph{Diagrams on surfaces and an invariant of virtual spatial graphs}.}

\newpage
\subsection{Expository}
\cvitem{Su2020}{UC Berkeley Lean seminar. $3$ talks about math in the Lean proof assistant.}
\cvitem{Fa2019}{Student $3$-manifold seminar, UCB. $6$ talks on topics in $3$-manifold topology.}
\cvitem{Sp2019}{Student $3$-manifold seminar, UCB. $8+$ talks on combinatorial $3$-manifold topology.}
\cvitem{Feb 2019}{$3$-manifold seminar, UCB. \emph{The arithmeticity of figure eight knot orbifolds}.}
\cvitem{Nov 2018}{$3$-manifold seminar, UCB. \emph{What is an alternating knot?}}
\cvitem{Sep 2018}{GRASP, UCB. \emph{The Jones polynomial and the Temperley--Lieb category}.}

\cvitem{Nov 2017}{Knot theory topics course, UCB. \emph{Quandles}.}
\cvitem{Sep 2017}{$3$-manifold seminar, UCB. \emph{Spatial graph invariants}.}
\cvitem{Apr 2017}{Knot Another Seminar, UCB. \emph{The Alexander ideal}.}
%\cvitem{Nov 2009}{MIT Splash. \emph{The Joy of Eigenvalues} and \emph{A Traversal of Graph Theory}.}


\section{Service}
\cvitem{2021--present}{Maintainer for \href{https://github.com/leanprover-community/mathlib}{\texttt{mathlib}}, the Lean mathematics library.}
\cvitem{Sep. 2023}{Co-instructor for \emph{Formal Mathematics and Computer-Assisted Proving} workshop at Hausdorff Center for Mathematics, University of Bonn}
\cvitem{June 2023}{Co-instructor and invited speaker for \emph{Formalization of Mathematics} workshop at SLMath (formerly MSRI) in Berkeley, CA}
%\cvitem{2020--present}{Contributor to \href{https://github.com/leanprover-community/mathlib}{\texttt{mathlib}}.}
\cvitem{2020}{Reviewed for Annales de l'Institut Henri Poincar\'{e} D: Combinatorics, Physics and their Interactions.}
\cventry{Fa2019}{Student $3$-Manifold Seminar (organizer)}{University of California}{Berkeley, CA}{}{}
\cventry{Sp2019}{Student $3$-Manifold Seminar (organizer)}{University of California}{Berkeley, CA}{}{}
\cventry{2015--2019}{Directed Reading Program (mentor)}{University of California}{Berkeley, CA}{}{
  Fall 2015, Spring 2017, Fall 2017, Fall 2018, Fall 2019.
  }

% \section{Workshops attended}

% \cvitem{Jun 2022}{Sage Days Duluth, to further development of SnapPy and Sage}


\section{Teaching Experience}

\subsection{University of California, Santa Cruz}

\cvitem{Wi2024}{Math 11B Calculus with Applications (174 students)}
\cvitem{Sp2022}{Math 116 Combinatorics}
\cvitem{Wi2022}{Math 110 Number Theory}

\subsection{University of California, Berkeley}

\cvitem{Fa2020}{Discussion sections, Math 54 Linear Algebra}
\cvitem{Sp2020}{Discussion sections, Math 1B Calculus}
\cvitem{Sp2017}{Discussion sections, Math 55 Discrete Mathematics}
\cvitem{Fa2016}{Discussion sections, Math 54 Linear Algebra}
\cvitem{Su2016}{Lecture and discussion sections, Math 54 Linear Algebra}
\cvitem{Sp2016}{Discussion sections, Math 54 Linear Algebra}
\cvitem{Fa2015}{Discussion sections, Math 1B Calculus}
\cvitem{Sp2015}{Discussion sections, Math 1A Calculus}
\cvitem{Fa2014}{Discussion sections, Math 1A Calculus}

\newpage
\section{Additional research experience}

\cventry{2009--2010}{UROP}{MIT Computer Science and AI Laboratory (CSAIL)}{Cambridge, MA}{}{
Worked on natural human-computer interactions for mathematics, and worked on expert systems for designing vehicles for a DARPA project. With Randall Davis.
}

\cventry{Sp2009}{UROP}{MIT Humans and Automation Laboratory}{Cambridge, MA}{}{
Developed a software platform for measuring the effects of team structures on situational awareness.
}

\section{Awards}

\cvitem{2018--2019}{Awarded support by the UCB NSF Research Training Group in Geometry and Topology for Spring 2018, Spring 2019, Summer 2019, and Fall 2019.}
\cvitem{2009}{MIT Licklider UROP prize for the best undergraduate research project in the area of human-computer interaction.}

\section{Personal}
\cvline{2011}{MIT Philip Loew Memorial Award for creative accomplishment in music.}



% \section{Research Experience}

% \cventry{Su2011}{UROP}{MIT Math Department}{Cambridge, MA}{}{
% Worked on software for four-dimensional visualization to study spherical codes. With Abhinav Kumar and Henry Cohn.
% }
% \cventry{Su2010}{UROP}{MIT Math Department}{Cambridge, MA}{}{
% Classified dynamics of $x\mapsto x^2\pmod{p}$. With Abhinav Kumar.
% }




% \section{Skills}
% \cvitem{Computer}{\emph{Programming:} Python, Haskell, Mathematica, Lean, JavaScript, Scheme, Common Lisp, C, C++, Java, C\#, J, Agda. \emph{Document preparation:} LaTeX, HTML/CSS.}


\end{document}
